\documentclass[UTF8]{ctexart}

\usepackage[linesnumbered,boxed,ruled,commentsnumbered]{algorithm2e}
\usepackage{bm}
\usepackage{graphicx}
\usepackage{float}
\usepackage[bookmarks=true]{hyperref}
\usepackage{amsmath}
\begin{document}
\title{人工智能综合作业3——MountainCar}
\author{陈昭熹 2017011552}
\maketitle
\tableofcontents
\newpage
\section{引言}
本次综合作业选择任务2,解决离散行动空间与连续行动空间下的小车上山问题(MountainCar)。对于两个问题一些共同的约定将在本节予以阐述。
\subsection{探索空间}
Agent可以感知的环境变量有小车当前位置以及小车速度,两个变量的探索空间定义如下:
\begin{table}[H]
    \centering
    \begin{tabular}{ccc}
        \hline
        Observation& 最小值& 最大值\\
        \hline
        Position & -1.2 & 0.6\\
        Velocity & -0.07 & 0.07\\
        \hline
    \end{tabular}
    \end{table}

\subsection{初始状态}
在本环境中,初始状态采用随机策略,取$[-0.6,-0.4]$的随机位置初始化agent状态,且速度为0。
\subsection{终止状态}
终止状态即小车到达预定山顶或迭代步数超过200,即:
\begin{equation}
    \textbf{Position} =0.5\;\; or \;\;\textbf{iteration} > 200 \rightarrow \textbf{terminate}
\end{equation}

\section{任务一 MountainCar-v0}
由于该问题属于无模型问题,采用基于行动价值,迭代更新Q表的训练方式比较合理。结合课内所学,采用Q-Learning, Sarsa, 期望Sarsa来解决这一问题。
\subsection{行动空间}
本任务的行动空间是离散的,仅有三个可选行动来控制小车的"油门":向左加速,向右加速与不加速。
\subsection{回报定义}
采用较为简单的前进代价式定义,即每走一步给予agent一个-1的回报,直到终止状态。这样的定义与gym库中定义相同,因此可以直接使用。
\begin{equation}
    reward \leftarrow reward - 1 ,\; for \;each\; step
\end{equation}

\subsection{$\epsilon$衰减}
为了权衡探索与利用,本文方法对于$\epsilon-greedy$策略中的$\epsilon$采取随着训练轮次衰减的策略,从1开始逐渐衰减到0,而非一开始就是一个极小值。这样的做法是为了让训练初期给予agent充分探索环境的机会($\epsilon$越大则越可能执行随机策略),这让训练后期回报的收敛提供必要的条件,避免局部最优解的产生。


\subsection{Q-Learning}\label{qlearning}
Q-Learning属于离线学习的控制方法,其行动策略遵循$\epsilon-greedy$原则,而目标策略则使用贪心策略进行选择。由此有其行动价值递推式:
\begin{equation}
    Q(S_t,A_t)\leftarrow Q(S_t,A_t) + \alpha(R_{t+1}+\gamma {max}_{a\in A}Q(S_{t+1},a)-Q(S_t,A_t))
\end{equation}

上式中的行动空间$A$与状态空间$S$定义及取值已经在前面的章节给出。在训练过程中,使用上式的方法更新行动价值,并反复迭代直至收敛,将最终的Q表存下来,用于回放训练结果。将前10000轮训练的平均收益、最大收益、最小收益与片段序号作折线图,得到训练过程的回报变化如下所示:
\begin{figure}[H]
    \centering
    \includegraphics[scale=0.6]{../disc_ql.jpg}
    \caption{任务一 Q-Learning回报收敛过程}
\end{figure}
\subsection{Sarsa}\label{sarsa}

Sarsa属于在线学习的控制方法,其行动策略与目标策略均遵循$\epsilon-greedy$原则,且行动策略与目标策略相同,由此产生行动价值递推式:
\begin{equation}
    Q(S_t,A_t)\leftarrow Q(S_t,A_t) + \alpha(R_{t+1}+\gamma Q(S_{t+1},a)-Q(S_t,A_t))
\end{equation}

在训练过程中,使用上式的方法更新行动价值,并反复迭代直至收敛,将最终的Q表存下来,用于回放训练结果。将前10000轮训练的平均收益、最大收益、最小收益与片段序号作折线图,得到训练过程的回报变化如下所示:
\begin{figure}[H]
    \centering
    \includegraphics[scale=0.6]{../disc_sarsa.jpg}
    \caption{任务一 Sarsa回报收敛过程}
\end{figure}
对比Q-Learning的回报收敛过程,可以发现Sarsa在平均回报上并无差别,但在最大回报上明显高于Q-Learning,这表征了Sarsa是一个更加保守的算法,相比于Q-Learning的用于探索,Sarsa更乐于保证状态的"安全",这一点在任务二\ref{sarsaexpriment}中更换不同的回报定义后会有明显的反映。同时也可以看到,Sarsa在最小回报的收敛上要明显早于Q-Learning,这也说明了Sarsa更保守,而Q-Learning则更倾向于探索。

\subsection{期望Sarsa}\label{expsarsa}
期望Sarsa也属于离线学习的控制方法,其行动策略遵循$\epsilon-greedy$原则,而目标策略则取期望意义上的估计值。可以理解为Sarsa是单点采样,而期望Sarsa则是均匀采样并计算期望。由此有其行动价值递推式:
\begin{equation}
    Q(S_t,A_t)\leftarrow Q(S_t,A_t) + \alpha(R_{t+1}+\gamma \sum_{a\in A} \pi(a|S_{t+1}) Q(S_{t+1},a)-Q(S_t,A_t))    
\end{equation}

在训练过程中,使用上式的方法更新行动价值,并反复迭代直至收敛,将最终的Q表存下来,用于回放训练结果。将前10000轮训练的平均收益、最大收益、最小收益与片段序号作折线图,得到训练过程的回报变化如下所示:
\begin{figure}[H]
    \centering
    \includegraphics[scale=0.6]{../disc_expsarsa.jpg}
    \caption{任务一 Expected Sarsa回报收敛过程}
\end{figure}

从上面的折线图中可以看出,与上面两种方法对比,无论从收敛速度上,从结果上,期望Sarsa均优于前面的两种方法。

\section{任务二 MountainCarContinuous-v0}
\subsection{行动空间}
本任务的行动空间是连续的,不仅可以决定加速的方向,还可以决定加速的大小,若用$A_i$表示状态$i$下的行动,则其取值应当是全体实数,用其符号区分加速方向,用其绝对值代表加速大小:
\begin{equation}
    A_i \in R,\begin{cases}
        push \; left , A_i<0\\
        push \; right, A_i > 0\\
        no \; push, A_i = 0\\
    \end{cases}
\end{equation}
\subsection{回报定义}
与任务一的回报定义不同,此处到达终点将给予agent回报100,并且计算此路径上花费的代价,因此累计回报可能是正值。
\begin{equation}
    reward=\begin{cases}
        100-\sum_{n}^{i=1}{A_i}^2, \; where\; state\; is\; terminated\\
        -\sum_{n}^{i=1}{A_i}^2,\; otherwise
    \end{cases}
\end{equation}
由于本任务中行动空间是连续的,因此不能直接用上一节所述的基于Q表的方法,或者说应当做一定处理才可以使用基于Q表的方法。本文最终给出基于离散行动空间的方法,求解时仍用与上一任务类似的训练算法,但是需要将行动空间进行合适的离散化。

\subsection{行动空间离散化}

为了仍使用基于Q表的诸多经典方法,在本问题中可以将行动空间离散化。只要行动空间的分割间隔合适,就不会影响连续行动空间下的小车运动,从而将连续问题转化为离散问题予以解决。经过多次实验验证,最终将本任务中的行动空间离散化为下面的形式:
\begin{equation}
    A_i \in \{-2.0,-1.6-1.2,-0.8,-0.4,0,0.4,0.8,1.2,1.6,2.0\}
\end{equation}

这样的处理可以大幅减少计算量,同时让上一章节中的算法可以继续使用。值得注意的是,这样的离散化考虑了物理约束条件,即小车坐标的范围在-1.2到0.6之间,因此行动空间内的最大值与最小值没必要过大,即小车不可能一次加速就冲出地图。

\subsection{行动空间离散化后的训练算法}
下面的三种算法在原理上与上一章节类似,因此只阐述在本任务中实现上的细节问题。下面的三种方法最终平均回报均收敛在90到100之间,符合官方wiki对于成功解决问题的定义。
\subsubsection{Q-Learning}
原理见\ref{qlearning}

用类似的方法将回报与片段序列绘制折线图如下所示:
\begin{figure}[H]
    \centering
    \includegraphics[scale=0.6]{../cont_ql.jpg}
    \caption{任务二 Q-Learning回报收敛过程}
\end{figure}

首先可以看出,本任务中回报的定义与上一任务不同,因此最终得到的回报结果会出现正值。而这样的回报定义虽然有些不寻常,但是从其收敛过程上可以看到,它能够帮助agent在短时间内更快的对环境建立正向认知,并朝着目标方向稳步努力(在前4000轮中平均回报近乎线性上升)。这是由于目标点的大额回报值能够更好的传递到前面的状态中,让智能体更好的感知到任务目标。
从其收敛后期可以看到,最小最大和平均收益几乎收敛到相同的值,而不像上一任务中仍存在较大差异。
\subsubsection{Sarsa}\label{sarsaexpriment}
原理见\ref{sarsa}
上文提到过,Sarsa是一个保守的方法,而在本任务的回报定义下,显然对于不善于探索的方法是不友好的,极有可能让这类方法陷入局部最优解,而无法及时发现目标状态。在最初的实验中,使用Sarsa方法就遇到了类似的问题,迭代10000轮后最终智能体认为停留在原点仿佛是回报最高的选择,得到其回报收敛过程如下:
\begin{figure}[H]
    \centering
    \includegraphics[scale=0.6]{../cont_sarsa_stupid.jpg}
    \caption{任务二 憨憨的Sarsa回报收敛过程}
\end{figure}

而相同的参数对于前一节的Q-Learning则不会出现陷入局部最优的问题。为此对于本任务中Sarsa的训练过程进行参数调整,增强其"探索性",将行动空间的范围扩大,同时缩减$\epsilon$衰减系数,让确定策略出现的更晚,并获得了如下结果:

\begin{figure}[H]
    \centering
    \includegraphics[scale=0.6]{../cont_sarsa.jpg}
    \caption{任务二 Sarsa回报收敛过程}
\end{figure}    

从这样的对比试验中,更能看出Sarsa与Q-Learning的区别,同时也凸显了回报的定义对于强化学习训练过程的重要影响。

\subsubsection{期望Sarsa}
原理见\ref{expsarsa}

用类似的方法将回报与片段序列绘制折线图如下所示:
\begin{figure}[H]
    \centering
    \includegraphics[scale=0.6]{../cont_expsarsa.jpg}
    \caption{任务二 Expected Sarsa回报收敛过程}
\end{figure}

从收敛的后期过程可以看出,期望Sarsa能够做到平均期望与最大期望基本一致,也就是说几乎每一个片段的选择都是最优的。而这样的性能是前面Q-Leanring和Sarsa所做不到的。在本任务中三种方法的收敛速度相同,而Expected Sarsa的性能显然更优。

\end{document}